\chapter{\ifcpe บทนำ\else Introduction\fi}

\section{\ifcpe ที่มาของโครงงาน\else Project rationale\fi}

กาแฟเป็นเครื่องดื่มที่คนเรานิมยมกันมากในปัจจุบัน โดยเริ่มมากจากร้านแฟเปิดขึ้นมากมาย ทำให้กลุ่มคนที่ชื่นชอบการดื่มกาแฟอย่างจริงจังมีการเติบโตมากขึ้น และกาแฟในประเทศไทยก็เริ่มมีความสนใจจากต่างชาติมากขึ้น จากการปลูกกาแฟของเกษตรภาคเหนือ และคนไทนหันมาดื่มกาแฟสดมากขึ้นทำให้ตลาดกาแฟนั้นเติบโตขึ้นไปอีก

ดั้งนั้นจากการเติบโตของตลาดกาแฟในประเทศไทยมีแนวโน้มการเติบโตมากขึ้น จึงจัดทำเว็บบล็อกที่จะรวบรวมบทความเกี่ยวกับกาแฟ และการแสดงความคิดเห็นหรือการรีวิวกาแฟสำหรับผู้ที่สนใจเกี่ยวกับกาแฟและผู้ที่สนใจเกี่ยวกับกาแฟอย่างจริงจัง ได้มีการนำเสนอเรื่องราวเกี่ยวกับกาแฟและเป็นแพลตฟอร์มที่ทำให้ผู้คนรู้จักเกี่ยวกับกาแฟมากขึ้น เช่น เกษตรกรที่ปลูกกาแฟได้นำเสนอกาแฟของตนเองและให้ความรู้เกี่ยวกับกาแฟให้กับผู้ที่สนใจ เพื่อให้ผู้คนรู้จักกาแฟ และขับเคลื่อนตลาดกาแฟในประเทศไทยและต่างประเทศ
\section{\ifcpe วัตถุประสงค์ของโครงงาน\else Objectives\fi}
\begin{enumerate}
    \item เพื่อทำเว็บที่รวบรวมบทความเกี่ยวกับกาแฟ
    \item เพื่อให้ผู้คนที่สนใจเกี่ยวกับกาแฟมาแลกเปลี่ยนประสบการณ์และความรู้เกี่ยวกับกาแฟ
    \item เพื่อให้ตลาดกาแหเติบโตขึ้นไปอีก จากการรวบรวมคนที่สนใจกาแฟเข้าด้วยกัน
\end{enumerate}

\section{\ifcpe ขอบเขตของโครงงาน\else Project scope\fi}

\subsection{\ifcpe ขอบเขตด้านฮาร์ดแวร์\else Hardware scope\fi}
\begin{enumerate}
    \item อุปกรณ์คอมพิวเตอร์สำหรับเขียน code และ test 
\end{enumerate}
\subsection{\ifcpe ขอบเขตด้านซอฟต์แวร์\else Software scope\fi}
\begin{enumerate}
    \item ฐานข้อมูลที่รองรับจำนวน users บทความ และรีวิว โดยจะสามารถใช้งานทั้งสมาร์ทโฟน แท็บเล็ต และคอมพิวเตอร์
\end{enumerate}
\section{\ifcpe ประโยชน์ที่ได้รับ\else Expected outcomes\fi}
\begin{enumerate}
    \item เป็นเว็บไซต์ที่รวบรวมบทความเกี่ยวกับกาแฟ
    \item เป็นเว็บไซต์ที่สามารถแบ่งบันความรู้เกี่ยวกับกาแฟได้
    \item เป็นเว็บไซต์ที่ทำให้วงการกาแฟเติบโตได้
\end{enumerate}

\section{\ifcpe เทคโนโลยีและเครื่องมือที่ใช้\else Technology and tools\fi}

\subsection{\ifcpe เทคโนโลยีด้านฮาร์ดแวร์\else Hardware technology\fi}
\begin{enumerate}
    \item Device: คอมพิวเตอร์ที่สามารถเปิดเว็บไซต์ได้และสามารถทำ responsive ได้
\end{enumerate}

\subsection{\ifcpe เทคโนโลยีด้านซอฟต์แวร์\else Software technology\fi}
แบ่งเป็นสองส่วน คือ front-end และ back-end โดย
\begin{enumerate}
    \item front-end ใช้ react next.js ในการทำเว็บไซต์
    \item back-end ใช้ elasticsearch และ mongodb ในการเก็บข้อมูล โดย elasticsearch ใช้ในการเก็บข้อมูลของการ search และ mongodb เก็บข้อมูลของบทความและรีวิว
\end{enumerate}
และใช้ Graphql เป็นตัวกลางในการเรียกใช้ข้อมูล

\section{\ifcpe แผนการดำเนินงาน\else Project plan\fi}

\begin{plan}{6}{2020}{2}{2021}
    \planitem{6}{2020}{7}{2020}{ศึกษาและค้นคว้า รวบรวมข้อมูลของกลุ่มเป้าหมาย}
    \planitem{7}{2020}{8}{2020}{วิเคราะห์ปัญหาและศึกษาความเป็นไปได้ของระบบ}
    \planitem{7}{2020}{10}{2020}{ศึกษาด้านการพัฒนาระบบ}
    \planitem{9}{2020}{112}{2020}{ดำเนินการทำงานพัฒนาระบบ}
    \planitem{1}{2021}{2}{2021}{ทดสอบและปรับปรุงระบบ} 
    \planitem{1}{2021}{2}{2021}{สรุปผลการทำงาน}
\end{plan}

\section{\ifcpe บทบาทและความรับผิดชอบ\else Roles and responsibilities\fi}
ในการทำงานที่ต้องทำงานคนเดียวจะต้องทำทั้งหมดด้วยตัวเอง ซึ่งจะต้องศึกษาการช้งาน เก็บข้อมูล ออกแบบหน้าเว็บไซต์ และเพื่อให้ใช้ในอุปกรณ์ต่างๆจะต้องออกแบบให้เป็น responsive ออกแบบระบบและทำงานทั้ง front-end และ back-end และทดสอบระบบ ซึ่งในการทำงานคนเดียวจะต้องจัดการเวลาและวางแผนการทำงานเผื่อให้สำเร็จตามเวลาที่กำหนด

\section{\ifcpe%
ผลกระทบด้านสังคม สุขภาพ ความปลอดภัย กฎหมาย และวัฒนธรรม
\else%
Impacts of this project on society, health, safety, legal, and cultural issues
\fi}
ในการทำโครงงานเผื่อให้เป็นที่รวบรวมข้อมูล บทความที่เกี่ยวกับกาแฟ และให้ผู้ที่สนใจเกี่ยวกับกาแฟได้มาแลกเปลี่ยนความรู้ ทำให้วงการกาแฟเติบโตและเป็นที่รู้จักมากขึ้น เช่น เกษตรกรที่ปลูกกาแฟได้มาเขียนบทความแลกเปลี่ยนความรู้และเป็นการนำเสนอเมล็ดกาแฟของตนเองด้วย หรือร้านกาแฟก็สามารถใช้เว็บไซต์นี้เขียนบทความเกี่ยวกับร้านตนเองให้มีความน่าสนใจ และเพื่อนำเสนอร้านให้เป็นที่รู้จักมากยิ่งขึ้น