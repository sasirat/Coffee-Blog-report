\chapter{\ifcpe ทฤษฎีที่เกี่ยวข้อง\else Background Knowledge and Theory\fi}

การทำโครงงาน เริ่มต้นด้วยการศึกษาค้นคว้า ทฤษฎีที่เกี่ยวข้อง หรือ งานวิจัย/โครงงาน ที่เคยมีผู้นำเสนอไว้แล้ว ซึ่งเนื้อหาในบทนี้ก็จะเกี่ยวกับการอธิบายถึงสิ่งที่เกี่ยวข้องกับโครงงาน เพื่อให้ผู้อ่านเข้าใจเนื้อหาในบทถัดๆ ไปได้ง่ายขึ้น

\section{ความสำคัญของกาแฟ}
กาแฟเป็นพืชเศรษฐกิจสำคัญประเภทหนึ่งซึ่ง เป็นสินค้าที่มีโอกาสและศักยภาพสูงในการทำตลาด โดยได้รับการจัดอันดับว่าเป็นพิชเศรษฐกิจที่มีมูลค่าสูง และนำไแแปรรูปให้มีมูลค่าสูงมากไปอีก โดยในประเทศไทยเป็นผู้ส่งออกกาแฟสำเร็จรูปอยู่ในอันดับที่ 11 ของโลกในปี พ.ศ. 2562 และในประเทศไทยคนส่วนดื่มกาแฟเผื่อบ่งบอกรสนิยมส่วนตัวด้วย โดยมีตัวเลขบ่งบอกว่าคนไทยดื่มกาแฟคนละ 300 แก้วต่อปี ซึ่งมีผลต่ออัตราการขยายตัวของตลาดกาแฟในประเทศไทย การมีร้านค่าเฟ่ที่ขายกาแฟที่มากขึ้นในประเทศไทยที่มีเอกลักษณ์และความสวยงามทำให้ผู้คนมีความสนใจที่จะไปร้านมากขึ้นและทำให้หลายๆคนได้ลองดื่มกาแฟแบบใหม่ๆ และผลเนื่องมาจาก Covid-19 ผู้คนมีความต้องการกาแฟสูงขึ้น ทำให้ผู้ประกอบการร้านกาแฟต้องทำแบบเดลิเวรี่ ซึ่งทำให้กาแฟสำเร็จรูปนั้นมีโอกาสในการส่งออกมากขึ้น

\section{Spacialty coffee}
Specialty Coffee คือ กาแฟพิเศษ ที่วัดกันตั้งแต่เมล็ดกาแฟจนถึง Process ทุกอย่างครับ เมล็ดกาแฟที่เป็น Specialty Coffee ต้องเป็นเมล็ดที่ชงออกมาแล้วผ่านกระบวนการคัด คั่ว บด กลั่น ชง จนได้กาแฟที่มีรสชาติดี ได้รับการรับรองว่ามีคุณภาพจากนักชิมที่มีความเชี่ยวชาญ ที่เรียกว่า Cupper หรือ Q – Grader โดยมีการทดสอบว่าในเรื่องกระบวนการผลิตเมล็ดกาแฟ การทดสอบคุณภาพ การทดสอบกลิ่นและรสชาติ และต้องได้คะแนน 80 คะแนนขึ้นไป ถึงจะเรียกว่า Specialty Coffee ได้ \cite{special}

\section{Coffee Taster's Flavor Wheel}
วงล้อกลิ่นและรสชาติกาแฟ เป็นเครื่องมือที่มีประโยชน์มากๆสำหรับนักชิมกาแฟ ในการวิเคราะห์และอธิบายกลิ่นและรสชาติกาแฟ เพราะบางครั้งมันก็เป็นอะไรที่ยากสำหรับมือใหม่ เนื่องจากการชิมกาแฟนั้น มีหลากหลายขั้นตอนที่ต้องทำเวลา และควบคุมอุณหภูมิ การจะวิเคราะห์รสชาติและกลิ่นนั้นอาจอยู่ภายใต้การจิบเร็วๆ เพียงไม่กี่ครั้ง แต่ ประโยชน์ที่แท้จริงของวงล้อสีนี้ คือเป็น Guideline สำคัญ ให้กับทั้งคนดื่ม และคนชงกาแฟ คุยกันรู้เรื่อง, ผมก็เลยแยกแบ่งเป็นกลุ่มหมวดต่างๆ 9 หมวด เพื่อให้มองเห็นชัด และเข้าใจได้ง่ายขึ้น \cite{wheel}

ซึ่งภายในเว็บไซต์ที่ทำจะให้การวิเคราะห์นี้ในการรีวิวกาแฟ เพื่อให้ผู้ใช้ได้มีความสนุกมากขึ้นในชิมกาแฟต่างๆ แบ่งปันรสชาติ และแสดงความคิดเห็นเกี่ยวกับรสชาติของกาแฟที่ตนเองได้ชิมให้ผู้ที่เข้ามาอ่านเห็นภาพและมีความน่าสนใจมากขึ้น

\section{การพัฒนาซอฟต์แวร์}
\subsection{React}
React เป็น JavaScript Library ที่ Facebook เป็นผู้พัฒนาโดยเปิดเป็น opem source โดยมี 3 concept หลักคือ Component, state และ Props โดยข้อดีของ React ก็คือมีเครื่อมือทำงานด้วยเยอะซึ่งเป็นประโยชน์ต่อการพัฒนามาก \cite{react}

\subsection{Next.js}
Next.js เป็น framwork ที่ใช้สำหรับสร้างเว็บ โดยใช้ react เพื่อให้กระบวนการง่ายขึ้น ซึ่งข้อดีของ Next.js คือ มี route ให้ และใช้ง่านกับ css แบบ Module หรือ CSS-in-JS \cite{nextjs}

\subsection{Graphql}
Graphql คือ ภาษาที่ใช้สำหรับเข้าถึงข้อมูล เพื่อใช้งาน API และประมวลผลคำสั่งในฝั่ง server \cite{graphql}

\subsection{node.js}
node.js คือ  Cross Platform Runtime Environment ของฝั่ง server โดยใช้ภาษา Javascript \cite{nodejs}

\subsection{Elasticsearch}
Elasticsearch เป็นเครื่องมือในการทำ search ซึ่งจะสามารถส่งข้อมูล JSON ไปยัง Elasticsearch โดย Elasticsearch จะจัดเก็บข้อมูลต้นฉบับโดยอัตโนมัติและอ้างอิงเอกสารในดัชนีของ cluster และสามารถดึงข้อมูลโดยใช้ Elasticsearch API ข้อดีก็คือสามารถดำเนินการแบบเรียลไทม์\cite{elastic}

\subsection{MongoDB}
MongoDB เป็น open source document database โดยจะเป็น Database แบบ NoSQL โดยจะไม่เน้นความสัมพันธ์ของข้อมูลและจัดเก็บข้อมูลแบบ JSON มีรูปแบบการจัดเก็บ คือ Collections และ schemaless ข้อดีคือไม่ต้องมีโครงสร้างของข้อมูล \cite{mongodb}

\subsection{การทำ Responsive design}
เป็นการออกแบบเว็บไซต์ให้รองรับกับหน้าจอของสมาร์ทโฟน แท็บเล็ต และหน้าจอคอมพิวเตอร์ เมื่อ user จะเข้าหน้าเว็บผ่านทางใดก็จะสามารถใช้งานได้อย่างปกติและมีความสวยความเข้ากับอุปกรณ์นั้นๆ 

\section{\ifcpe%
ความรู้ตามหลักสูตรซึ่งถูกนำมาใช้หรือบูรณาการในโครงงาน
\else%
ISNE knowledge used, applied, or integrated in this project
\fi
}
\begin{enumerate}
    \item การออกแบฐานข้อมูล โดยใช้ความรู้จากวิชา FUND OF DATABASE SYSTEMS รหัส
วิชา 261342 มาใช้ในการพัฒนาซอฟแวร์ด้วย MongoDB
    \item กระบวนการในการพัฒนาซอฟต์แวร์ นำความรู้มาจากวิชา SOFTWARE
ENGINEERING รหัสวิชา 261361 โดยใช้ในการวาแผนการทำงานต่างๆ
    \item การออกแบบการใช้งานของ uer จากวิชา Human-Computer Interaction รหัสวิชา 269462 ใช้ความรู้ในการออกแบบให้ผู้ใช้สามารถใช้งานได้ง่าย
\end{enumerate}

\section{\ifcpe%
ความรู้นอกหลักสูตรซึ่งถูกนำมาใช้หรือบูรณาการในโครงงาน
\else%
Extracurricular knowledge used, applied, or integrated in this project
\fi
}

การทำงานสิ่งที่ต้องเรียนรู้ด้วยตนเอง คือ การเรียนรู้เกี่ยวกับการทำฐานข้อมูลเกี่ยวกับการ search และเรียนรู้การออกแบบหน้าเว็บที่ทำให้ผู้ใช้เข้าใจง่ายและมีความสวยงาม 
